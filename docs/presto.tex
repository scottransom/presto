\documentclass[11pt]{article}

\newcommand{\fdot}{$\dot{f}$}
\newcommand{\ffdot}{$f$--$\dot{f}$}
\newcommand{\PRESTO}{{\tt PRESTO}}

\title{\huge \PRESTO\thanks{That's the \emph{PulsaR Exploration and Search TOolkit}, according to Steve Eikenberry.}\\
  {\large Users Manual, Version 1.0}}
\author{Scott Ransom}

\begin{document}

\maketitle

\section{Introduction}

\PRESTO\ is a large suite of pulsar search and analysis software
developed almost from scratch and written primarily in ANSI C.
Written with portability, ease-of-use, and memory efficiency in mind,
it can currently handle data from the Berkeley-Caltech Pulsar Machine
(BCPM; located at the GBT), the Wideband Arecibo Pulsar
Processor (WAPP), the Parkes Multibeam system, or any individual time
series composed of single precision floating point data.  This last
format is useful for analyzing X-ray, optical, or dedispersed radio
data from any source.  Additional data formats such as phased-array
data from the GMRT and the SPIGOT card from the GBT are in the works.

The software is composed of numerous routines designed to handle three
main areas of pulsar analysis:
\begin{enumerate}
\item Data Preparation:  Interference detection and removal, de-dispersion, barycentering.
\item Searching:  Acceleration and phase-modulation (or sideband) searches.
\item Folding:  Candidate optimization and TOA generation.
\end{enumerate}
Additional utilities are provided for various tasks that are often
required when working with pulsar data such as time conversions, time
series and FFT exploration, byte-swapping, etc.

The intent of this manual is to provide an overview of \PRESTO's
capabilities, describe in some details the functioning of each of the
main programs, and to present various simple ``recipes'' that can be
used to accomplish typical pulsar analysis tasks.


\begin{enumerate}
\item Data Preparation
  \begin{enumerate}
  \item {\bf\tt rfifind} searches the raw data in both the time and
    frequency domains for interference or other problems.  Each
    channel is analyzed for short time intervals throughout the
    observation.  The result is a list of recommended portions of the
    data to mask.
  \item {\bf\tt prepsubband} clips, masks, dedisperses, barycenters,
    and pads raw data into numerous time series over a range of trial
    DMs.
  \item {\bf\tt realfft} performs either an in-core or out-of-core FFT
    based on the length of the time series to be transformed.  The
    input data is not required to be a power-of-two in length.
  \item {\bf\tt zapbirds} removes known sources of interference (or
    strong pulsars) from a Fourier transform by replacing them with
    the local average Fourier amplitudes.
  \end{enumerate}
\item Searching
  \begin{enumerate}
  \item {\bf\tt accelsearch} performs a Fourier domain acceleration
    search with Fourier interpolation and harmonic summing on an FFT.
    The code is written such that it uses a minimum of memory
    ($\sim200$\,MB for a search over 200 independent accelerations no
    matter what the length of the FFT) and utilizes the processor
    cache as much as possible.
  \item {\bf\tt search\_bin} performs a phase modulation sideband
    search on an FFT including harmonic summing and Fourier
    interpolation.
  \end{enumerate}
\item Candidate Optimization
  \begin{enumerate}
  \item {\bf\tt prepfold} folds known pulsars or candidates from {\tt
      accelsearch} over a range of DMs, periods, and period
    derivatives around the ``best-guess'' values and returns the
    optimized pulse profile.  All plots of new and candidate pulsars
    in this chapter were generated using {\tt prepfold}.
  \item {\bf\tt bincand} attempts to find an orbital solution for a
    sideband search candidate uncovered with {\tt search\_bin}.  Brute
    force matched filtering is used to search the orbital parameters
    around the ``best-guess'' values as discussed in
    chapter~\ref{chapter3}.
  \end{enumerate}
\end{enumerate}
Many other utilities for displaying, plotting, and working with data
and have been written in C or as Python\footnote{A free and
  aesthetically pleasing scripting language available from {\tt
    http://www.python.org}} scripts.  Since the software is coded in a
modular form and is relatively mature, adding new capabilities, such
as the ability to handle a new pulsar back-end, can be accomplished in
only a few hours of programming.  Similarly, with a large library of
routines that implement the ``building-block'' functionality of
Fourier-domain searches (such as complex correlations and template
generation), new searches or analysis techniques can be coded quickly
and efficiently.

\end{document}
